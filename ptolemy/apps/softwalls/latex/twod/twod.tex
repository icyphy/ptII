\documentclass[11pt]{article}
\usepackage{times}
\usepackage{ifthen}
\usepackage[dvips]{graphics}
\usepackage[dvips]{color}
\usepackage{epsfig}
%\usepackage{subfigure}
\usepackage[dvips,colorlinks,bookmarks,pdfpagemode=UseOutlines,linkcolor=blue,pagecolor=blue,urlcolor=blue,letterpaper]{hyperref}
\begin{document}
% File giving definitions of commands and environments.
% Because of a latex bug with counters, this has to be included
% using \input not \include.

%--------------------------------------------------------------------------
% For the set of reals and integers
\newcommand{\rr}{\set{Reals}}
\newcommand{\ii}{\set{Integers}}
\newcommand{\cc}{\set{Complex}}
\newcommand{\nn}{\set{Naturals}}

%--------------------------------------------------------------------------
% For figure captions.
% Puts them in sanserif font, indented on both sides.
%   arg: The caption.
\newcommand{\figcaption}[1]{\textsf{\begin{center}\begin{minipage}{5in}
\caption{#1}
\end{minipage}\end{center}}}

%--------------------------------------------------------------------------
% For terms being defined.
% Puts them in bold face and creates an index entry.
%   arg: The term being defined.
% NOTE: To get boldface in the index, do |textbf after #1.
% But this breaks hyperlinks.
\newcommand{\defn}[1]{\textbf{#1}\index{#1}}

%--------------------------------------------------------------------------
% For terms being indexed.
% Puts them in standard font face and creates an index entry.
%   arg: The term being defined.
\newcommand{\pointer}[1]{#1\index{#1}}

%--------------------------------------------------------------------------
% For bold terms to be index, but defined elsewhere
% Puts them in bold face and creates an index entry.
%   arg: The term being defined.
\newcommand{\strong}[1]{\textbf{#1}\index{#1}}

%--------------------------------------------------------------------------
% For terms to be index, but defined elsewhere
% Puts them in normal face and creates an index entry.
%   arg: The term being defined.
\newcommand{\idx}[1]{#1\index{#1}}

%--------------------------------------------------------------------------
% For set names.
% Puts them in italics. In math mode, yields decent spacing.
%   arg: The name of the set.
\newcommand{\set}[1]{\mbox{\textit{#1}}}

%--------------------------------------------------------------------------
% For real part.
%   arg: The argument of the real part.
\newcommand{\re}[1]{\mbox{\textit{Re}}\{#1\}}

%--------------------------------------------------------------------------
% For imaginary part.
%   arg: The argument of the imaginary part.
\newcommand{\im}[1]{\mbox{\textit{Im}}\{#1\}}

%--------------------------------------------------------------------------
% For matlab commands
%   arg: The name of the command
\newcommand{\matlab}[1]{\texttt{#1}\index{#1 command in Matlab}\index{Matlab!#1}}
\newcommand{\simulink}[1]{\texttt{#1}\index{#1 in Simulink}\index{Simulink!#1}}
\newcommand{\matlabInCaption}[1]{\texttt{#1}}

%--------------------------------------------------------------------------
% For "Probing Further" sidebars.
% Puts them in a floating frame.  It is up to you to ensure that the
% frame fits on one page.
%   arg: the title.
\newenvironment{further}[1]{
\begin{table}[btp]
\centering
\begin{tabular}{|p{5in}|}
\hline
\cr
\begin{minipage} {5in}
\parskip        0.1in
\parindent      0.0in
\subsection* {Probing further: #1}
\addcontentsline{toc}{subsection}{Probing further: #1}
} {
\end{minipage}\cr
\cr
\hline
\end{tabular}
\end{table}
}

%--------------------------------------------------------------------------
% For "Basics" sidebars.
% Puts them in a floating frame.  It is up to you to ensure that the
% frame fits on one page.
%   arg: the title.
\newenvironment{basics}[1]{
\begin{table}[btp]
\centering
\begin{tabular}{|p{5in}|}
\hline
\cr
\begin{minipage} {5in}
\parskip        0.1in
\parindent      0.0in
\subsection* {Basics: #1}
\addcontentsline{toc}{subsection}{Basics: #1}
} {
\end{minipage}\cr
\cr
\hline
\end{tabular}
\end{table}
}

%--------------------------------------------------------------------------
% For "Tips and Tricks" sidebars.
% Puts them in a floating frame.  It is up to you to ensure that the
% frame fits on one page.
%   arg: the title.
\newenvironment{tricks}[1]{
\begin{table}[btp]
\centering
\begin{tabular}{|p{5in}|}
\hline
\cr
\begin{minipage} {5in}
\parskip        0.1in
\parindent      0.0in
\subsection* {Tips and Tricks: #1}
\addcontentsline{toc}{subsection}{Tips and Tricks: #1}
} {
\end{minipage}\cr
\cr
\hline
\end{tabular}
\end{table}
}

%--------------------------------------------------------------------------
% For text that is boxed for emphasis.
\newenvironment{boxed}{
\begin{center}
\begin{tabular}{|p{5in}|}
\hline
\cr
\begin{minipage} {5in}
\parskip        0.1in
\parindent      0.0in
} {
\end{minipage}\cr
\cr
\hline
\end{tabular}
\end{center}
}

%--------------------------------------------------------------------------
% For examples
% NOTE: Because of this line, this has to be included using \input
% not \include.
\newcounter{example}
\newenvironment{example}{
\refstepcounter{example}
\begin{quote}
\textbf{Example \arabic{chapter}.\arabic{example}:}
} {
\end{quote}
}
\renewcommand \theexample {\thechapter.\arabic{example}}


\title{SoftWalls in 2D}
\author{J. Adam Cataldo, Edward A. Lee, Ashwin Ganesan, and
Stephen Neuendorffer\\ \{acataldo, eal, ganesan, neuendor\}@eecs.berkeley.edu\\ Electrical
Engineering \& Computer Science\\ University of California, Berkeley\\
\\} 
\date{24 June 2002}

\maketitle

%-----------------------------------------------------------------------------
%-----------------------------------------------------------------------------

\section{Introduction}

This document gives an outline of a general 2D softwalls approach.  We
assume the initial speed is known.

%-----------------------------------------------------------------------------
%-----------------------------------------------------------------------------


\section{Step Bias}

%-----------------------------------------------------------------------------

Suppose the aircraft is travelling at a constant speed $s$.  Let
$\theta$ denote the aircraft's heading angle, and let
$\dot{\theta_{p}}$ be the pilot's control input, i.e. the desired rate
of change in heading angle.  At that speed, the aircraft has a
minimum-safe turning radius $r_{min}$.  This contrains the paths at
which the aircraft can bank right or left as in figure \ref{paths}.

\begin{figure}[btp]
\centering
\includegraphics[width=5in]{aircraftpaths.eps}
\figcaption{The dot represents the aircraft, which is moving along the
center path.  The arc paths represtent turns at the minimum turning radius.
\label{paths}}
\end{figure}

Now suppose we wish to prevent the aircraft from crossing a boundry.
As the aircraft approaches the boundry, one of minimum-turning-radius
paths will interesect with the boundry.  As long as the other path has
not yet interestected the boundry, the aircraft can still avoid
crossing the boundry by turning along this path.  When the other path
intersects the boundry, turning along this path until one of the paths
no longer intersects the boundry will prevent the aircraft from
crossing the boundry.

If the actual rate of change in aircraft heading angle be
$\dot{\theta}$, and we limit this value to the safe range
$[-s/r_{min}, s/r_{min}]$, we can view this controller as follows:

Let $\theta_{s}$, be the softwalls-generated control signal.  We
calculate $\dot{\theta}$ from
\[
\dot{\theta} = limit_{[-s/r_{min}, s/r_{min}]}(\dot{\theta_{p}} -
\dot{\theta_{s}},
\]
where
\[
\set{limit}_{[a,b]}(u) = \left \{
\begin{array}{ll}
b&\mbox{ if } u > b,\\
a&\mbox{ if } u < a,\\
u&\mbox{ otherwise. }
\end{array}
\right .
\]

Assuming $\dot{\theta_{p}}$ is limited to $[-\dot{\theta_{M}},
\dot{\theta_{M}}]$, which corresponds to the limits of the flight
yoke, we can choose $\dot{\theta_{s}} = \dot{\theta_{M}} + s/r_{min}$
to force the aircraft to turn right at the maximum turning radius,
irrespective of pilot input.  Similarly $\dot{\theta_{s}} =
-\dot{\theta_{M}} - s/r_{min}$ will force the aircraft to turn left.

By applying the appropriate $\theta_{s}$ to the input, the softwalls
algorithm (the path/boundry-interestection algorithm) can garauntee the
plane to never crosses the boundry.  

\end{document}

